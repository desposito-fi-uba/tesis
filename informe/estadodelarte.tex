\section{Estado del arte}

En el presente capitulo se analizará el estado del arte de los canceladores de ruido para señales de habla captadas por un solo micrófono.

Uno de los primeros trabajos donde se estudió los canceladores de ruido con filtros basados en redes neuronales fue en; \emph{A regression approach to speech enhacement based on deep neuronal networks} \cite{a_regression_approach_to_speech_enhancement_based_on_deep_neural_networks}. Los autores del articulo son; Yong Xu, Jun Du, Li-Rong Dai, and Chin-Hui Lee y fue publicado en el año 2015.

La arquitectura de red neuronal usada es una red del tipo \emph{feedforward} completamente conectada con hasta 4 capas ocultas y cada una con hasta 2048 unidades. La base de datos usada para entrenar la red contiene 104 tipos de ruidos diferentes y un total de 100 horas de entrenamiento.

Una de las métricas usadas fue la PESQ y a continuación se detallan los valores que obtuvieron:

\begin{table}[H]
	\centering
	\begin{tabular}{ |c|c|c|c|c|c|c|c| } 
		\hline
		SNR [dB] & $-5$ & $0$ & $5$ & $10$ & $15$ & $20$ & Media \\ 
		\hline
		PESQ - Ruidosa & 1.37 & 1.61 & 1.90 & 2.22 & 2.55 & 2.88 & 2.09 \\
		PESQ - Filtrada & 1.75 & 2.09 & 2.44 & 2.78 & 3.09 & 3.37 & 2.59 \\
		STOI - Ruidosa & - & - & - & - & - & - & - \\
		STOI - Filtrada & - & - & - & - & - & - & - \\
		\hline
	\end{tabular}
	\caption{PESQ en función de la SNR.}
\end{table}

Otro de los trabajos interesantes para analizar es \emph{Speech Enhancement In Multiple-Noise Conditions using Deep Neural Networks} publicado por Anurag Kumar y Dinei Florencio en el año 2016 \cite{speech_enhancement_in_multiple_moise_conditions_using_deep_neural_networks}.

Nuevamente la arquitectura utilizada es del tipo feedforward completamente conectada con 3 unidades ocultas, cada una con 2048 unidades. La base de datos usada para entrenar la red tiene una duración total de 100 horas. 

Las medidas usadas para analizar los resultaron fueron la PESQ y la STOI. A continuación podemos encontrar los resultados que obtuvieron:

\begin{table}[H]
	\centering
	\begin{tabular}{ |c|c|c|c|c|c|c|c| } 
		\hline
		SNR [dB] & $-5$ & $0$ & $5$ & $10$ & $15$ & $20$ & Media \\ 
		\hline
		PESQ - Ruidosa & 1.46 & 1.77 & 2.11 & 2.53 & 2.88 & 3.23 & - \\
		PESQ - Filtrada & 1.92 & 2.32 & 2.69 & 3.09 & 3.40 & 3.67 & - \\
		STOI - Ruidosa & 0.612 & 0.714 & 0.813 & 0.898 & 0.945 & 0.974 & - \\
		STOI - Filtrada & 0.703 & 0.804 & 0.872 & 0.923 & 0.950 & 0.965 & - \\
		\hline
	\end{tabular}
	\caption{PESQ y STOI en función de la SNR.}
\end{table}

%Delta PESQ 0.5, 0.59, 0.63, 0.61, 0.56, 0.48
%Delta STOI 0.105, 0.098, 0.066, 0.03, 0.008, -0.004

El siguiente trabajo que resultó relevante de analizar fue uno de los primeros en proponer otro tipo de arquitectura de red neuronal. El nombre del artículo es \emph{A fully convolutional neural network for speech enhancement} publicado por Se Rim Park y Jinwon Lee en el año 2016 \cite{a_fully_convolutional_neural_network_for_speech_enhancement}. 

La arquitectura utilizada fue del tipo convolucional con 16 capas. La red contaba con un codificador y un decodificador similar a los descritos en \ref{sec:redes_autocodificadoras}. La base de datos usada para entrenar la red contiene 27 tipos de ruidos diferentes mezclados con las señales de habla únicamente a $\SI{0}{dB}$. A continuación podemos ver los resultados que obtuvieron:

\begin{table}[H]
	\centering
	\begin{tabular}{ |c|c|c|c|c|c|c|c| } 
		\hline
		SNR [dB] & $-5$ & $0$ & $5$ & $10$ & $15$ & $20$ & Media \\ 
		\hline
		PESQ - Ruidosa & - & - & - & - & - & - & - \\
		PESQ - Filtrada & - & 2.34 & - & - & - & - & - \\
		STOI - Ruidosa & - & - & - & - & - & - & - \\
		STOI - Filtrada & - & 0.83 & - & - & - & - & - \\
		\hline
	\end{tabular}
	\caption{PESQ y STOI en función de la SNR.}
\end{table}


Otro de los artículos analizados fue \emph{A Convolutional Recurrent Neural Network for Real-Time Speech} el cual utiliza una arquitectura convolucional con una capa recurrente. Ademas, lo interesante es que propone un procesamiento de las señales de manera causal para que pueda utilizarse para filtrado de ruido en tiempo real. El trabajo fue publicado por Ke Tan y DeLiang Wang en el año 2018 \cite{a_convolutional_recurrent_neural_network_for_real_time_speech_enhancement}.

La base de datos usada tiene una duración total de 126 horas y utilizan niveles de ruido en el rango [-5, 0] dB. A continuación podemos ver los resultados que obtuvieron:

\begin{table}[H]
	\centering
	\begin{tabular}{ |c|c|c|c|c|c|c|c|c| } 
		\hline
		SNR [dB] & $-5$ & $-2$ & $0$ & $5$ & $10$ & $15$ & $20$ & Media \\ 
		\hline
		PESQ - Ruidosa & 1.50 & 1.67 & - & - & - & - & - & - \\
		PESQ - Filtrada & 2.15 & 2.41 & - & - & - & - & - & - \\
		STOI - Ruidosa & 0.581 & 0.657 & - & - & - & - & - & - \\
		STOI - Filtrada & 0.778 & 0.840 & - & - & - & - & - & - \\
		\hline
	\end{tabular}
	\caption{PESQ y STOI en función de la SNR.}
\end{table}

WEIGHTED SPEECH DISTORTION LOSSES FOR
NEURAL-NETWORK-BASED REAL-TIME SPEECH ENHANCEMENT

Rango de SNRs: 0 a 40
14 tipos de ruidos
Usan una red basada en layers GRU
Usan una función de error compleja

Noisy PESQ 2.22
Noisy STOI 88

Filtered PESQ 2.65
Filtered STOI 90.7

Towards speech enhancement using a variational U-Net architecture

Rango de SNRs: 0 a 20

Cantidad de tipos de ruidos sin especificar

15 horas de entrenamiento

noisy pesq 2.52 
noisy stoi 91.51

filtered pesq 3.01 
filtered stoi 94.08

100 horas de entrenamiento

filtered pesq 3.11
filtered stoi 94.60

noisy pesq 2.52 
noisy stoi 91.51

A Perceptually-Motivated Approach for Low-Complexity, Real-Time Enhancement of Fullband Speech

Rango de SNRs: -5 a 45

Cantidad de tipos de ruidos sin especificar

120 horas de entrenamiento

filtered pesq 2.54

noisy pesq 1.97 



Feature enhancement by bidirectional LSTM networks for conversational speech recognition in highly non-stationary noise





Mi tesis 1

Rango de SNRs: -5 a 40
14 tipos de ruidos
Convolutional network

Noisy PESQ: 2.48
Filtered PESQ: 2.55

Noisy STOI: 0.89
Filtered STOI: 0.88

Mi tesis 2

Rango de SNRs: -5 a 20
14 tipos de ruidos
Convolutional network

Noisy PESQ: 2.20
Filtered PESQ: 2.40

Noisy STOI: 0.875
Filtered STOI: 0.871


