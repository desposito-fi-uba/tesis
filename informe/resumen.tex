\section*{\hfil Resumen \hfil}

\noindent El filtrado de ruidos no-estacionarios presentes en señales de habla es un problema que en los últimos años atrajo gran atención debido al proceso de digitalización acelerada que está sufriendo muchas de las actividades que el ser humano hace diariamente. Antes del gran desarrollo que tuvieron las redes neuronales profundas, el marco general para el filtrado de ruido no-estacionarios era por medio de la utilización de filtros adaptativos. El objetivo de esta tesis será comparar el filtrado de ruidos no-estacionarios por medio de filtros adaptativos y por medio de filtros neuronales, utilizando métricas objetivas como la medida de calidad PESQ y la medida de inteligibilidad STOI. Los resultados encontrados en este trabajo, sugieren que en condiciones de bajo nivel de ruido, un filtro neuronal es capaz de igualar el desempeño de un filtro adaptativo o incluso mejorarlo.